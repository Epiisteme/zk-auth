\documentclass[sigconf]{acmart}

\settopmatter{printacmref=false}
\setcopyright{none}
\renewcommand\footnotetextcopyrightpermission[1]{}
\pagestyle{plain}
\hypersetup{
    colorlinks=true,
    linkcolor=blue,
    filecolor=magenta,
    urlcolor=cyan,
}

\title{CNS Term Project Proposal}
\subtitle{A Deep Dive Into Anonymous Authentication System With Zero-Knowledge Proof}
\author{
    R08922161 \and
    M10915072 \and
    B08902124 \and
    B05901044
}

\begin{document}

\maketitle

\section{Motivation}
Since more and more people have got COVID-19 vaccine, the idea of vaccine
certificate system is mentioned in many place, which is said to help people
travel across country border easier. Although building such a large system
seems to be hard, we might expect the idea would be adopted in smaller scale
such as getting into workspace or theater without lining up taking temperature
in the near future. \\
However, in such use case, there is a main drawback that when we are proving
our ownership of a vaccine certificate, we might leak our personal information
to the verifier or the system, which is undesired invasion of privacy. So we
are wondering how to achieve a system able to authenticate that an user is in
a membership set without revealing his identity, which is also known as
\textbf{Anonymous Authentication System}. Then we found Zero-Knowledge Proof
might be the most important block to build such a system, which allows us prove
the possess of knowledge without revealing it. \\
Fortunately, ZKP is getting pratical for general programing since \cite{GGPR13}
and developer freindly with many tools showing up. So it might be worthwhile to
not only do the research of such system but also try to implement one with
modern tools.

\section{Plan}
To have a better understanding of current state of research about anonymous
authentication system, we will first do a comprehensive research on it to
check what's the possible solution to build such a system and what's the main
challenges we are facing. \\
In the meantime, we are going to understand how the state-of-art ZKP system
\cite{Gro16} works and try to build programmable ZKP above it with
\href{https://github.com/iden3/circom}{Circom}, which is a programming language
that allows us to build general program in ZKP easily. \\
Finally, we expect to implement an anonymous authentication system with ZKP
and then build a vaccine certificate system above it to show how it might help
in real world situation.

\section{Timeline}
We are going to have a checkpoint per week, and here is our goal to check in
each checkpoint:
\begin{enumerate}
    \item \textbf{05/11} - Having understanding about current solutions and
    problems about anonymous authenticate system.
    \item \textbf{05/18} - Having understanding about how \cite{Gro16} works and
    able to build general program in
    \href{https://github.com/iden3/circom}{Circom}.
    \item \textbf{05/25} - Finish the specification of the anonymous
    authenticate system and vaccine certificate system we are going to build.
    \item \textbf{06/01} - Finish anonymous authenticate system.
    \item \textbf{06/08} - Finish vaccine certificate system and TUI client to
    interact with it.
\end{enumerate}

\section{Deliverables}
In the end of this term project, we expect to deliver:
\begin{itemize}
    \item A research report about anonymous authentication system
    \item A proof-of-concept anonymous authentication system built with ZK-SNARK
\end{itemize}

\bibliographystyle{alpha}
\bibliography{references}

\end{document}
